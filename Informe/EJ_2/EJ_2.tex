\section{Ejercicio 2}

% El desarrollo de la primera expresión de 5 variables
% usando mintérminos se realiza en esta subsección
\subsection{Expresi\'on en mint\'erminos}
La subsecci\'on presente tratar\'a la siguiente expresi\'on caracterizada como:
\begin{equation*}
    f(e,d,c,b,a) = \sum m(0,2,4,7,8,10,12,16,18,20,23,24,25,26,27,28)
\end{equation*}


\subsubsection{Simplificación con Álgebra booleana}
En el siguiente desarrollo primero se escribe de forma completa la suma de productos que se describe en la expresi\'on
consignada, donde cada producto se compone de la combinaci\'on de entradas en cada estado indicado, de forma tal que el resultado 
de tal producto sea un estado l\'ogico activo, esto es, un '1' binario. Con el objetivo de proveer la mayor claridad posible en el desarrollo,
se subindican y supraindican los t\'erminos, en donde el sub\'indice describe la identificaci\'on de un t\'ermino y luego el supra\'indice describe
el conjunto de t\'erminos de los cu\'ales deriva.

\begin{equation*}
\begin{align*}
f(e,d,c,b,a) & = {\overline{e} \cdot \overline{d} \cdot \overline{c} \cdot \overline{b} \cdot \overline{a}}
+ {\overline{e} \cdot \overline{d} \cdot \overline{c} \cdot b \cdot \overline{a}} 
+ {\overline{e} \cdot \overline{d}  \cdot c \cdot \overline{b} \cdot \overline{a}}  
+ {\overline{e} \cdot d \cdot \overline{c} \cdot \overline{b} \cdot \overline{a}} 
+ {\overline{e} \cdot \overline{d} \cdot c \cdot b \cdot a}
+ {\overline{e} \cdot d \cdot \overline{c} \cdot b \cdot \overline{a}}  \\
& + {\overline{e} \cdot d \cdot c \cdot \overline{b} \cdot \overline{a}}
+ {e \cdot \overline{d} \cdot \overline{c} \cdot \overline{b} \cdot \overline{a}} 
+ {e \cdot \overline{d} \cdot \overline{c} \cdot b \cdot \overline{a}}
+ {e \cdot \overline{d} \cdot c \cdot \overline{b} \cdot \overline{a}}
+ {e \cdot \overline{d} \cdot c \cdot b \cdot a}
+ {e \cdot d \cdot \overline{c} \cdot \overline{b} \cdot \overline{a}} \\
& + {e \cdot d \cdot \overline{c} \cdot \overline{b} \cdot a} 
+ {e \cdot d \cdot \overline{c} \cdot b \cdot \overline{a}} 
+ {e \cdot d \cdot \overline{c} \cdot b \cdot a}
+ {e \cdot d \cdot c \cdot \overline{b} \cdot \overline{a}}
\end{align*}
\end{equation*}

\subsubsection{Simplificación con mapas de Karnaugh}

\subsubsection{Expresión desarrollada usando NOR}

\subsubsection{Circuitos lógicos}

% El desarrollo de la segunda expresión de 4 variables
% usando maxtérminos se realiza en esta subsección
\subsection{Expresi\'on en maxt\'erminos}

\subsubsection{Simplificación con Álgebra booleana}

\subsubsection{Simplificación con mapas de Karnaugh}

\subsubsection{Expresión desarrollada usando NOR}

\subsubsection{Circuitos lógicos}
