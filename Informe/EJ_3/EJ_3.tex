\section{Ejercicio 3}
De los dos módulos pedidos para implementar en Verilog se tomó la decisión de realizar uno mediante las compuertas lógicas que lo componen, y otro a través de la descripción de su comportamiento (behavioural).
La razón detrás de esta decisión fue para hacer uso de las variantes provistas por Verilog, e interiorizarnos en su estilo de programación.

\subsection{Decoder de 4 salidas}
Esta implementación de un decoder de 4 salidas recibe una arreglo de 2 bits, que determinan cual de las cuatro salidas accionar.
La misma se realizó a través de lógica de compuertas, donde únicamente la combinación correcta de los dos bits de entrada, ponen a la salida con ese número, en 1 lógico.
Las relaciones lógicas son sencillas de comprender y quedan explicitadas en el mismo código.
A continuación se presenta el código en Verilog:
\begin{lstlisting}
    module decoder4out(coded, y0, y1, y2, y3);
        input [1:0] coded;
        output y0, y1, y2, y3;

        assign y0 = ~coded[1] & ~coded[0];
        assign y1 = ~coded[1] & coded[0];
        assign y2 = coded[1] & ~coded[0];
        assign y3 = coded[1] & coded[0];

    endmodule
\end{lstlisting}

\subsection{MUX de 4 entradas}
Esta implementación de un mux de 4 entradas recibe un arreglo de 4 bits con las 4 "fuentes", otro arreglo de 2 bits que hace las veces de selector, y cuenta con una salida de 1 bit.
La salida copiará el valor de la entrada determinada por el selector.
El módulo se logró mediante una descripción del comportamiento del mismo, en el cual se le especificó qué debía realizar ante cambios en alguna de sus entradas.
A continuación se presenta el código en Verilog:
\begin{lstlisting}
    module mux4in (x, sel, y);
        input [3:0] x;
        input [1:0] sel;                      // sel selects the exit (x[3], x[2], x[1], x[0]).
        output reg y;

        always @(sel or x) begin
            if (sel == 0)
                assign y = x[0];
            else if (sel == 1)
                assign y = x[1];
            else if (sel == 2)
                assign y = x[2];
            else if (sel == 3)
                assign y = x[3];

        end

    endmodule
\end{lstlisting}

\subsection{Test-bench}
Se sometió a los dos módulos a un testeo de su respuesta a cada una de las posibles entradas, y los resultados fueron los esperados.
Los mismos pueden ser replicados ejecutando los comandos:
\begin{lstlisting}[language=bash]
    user@computer: path/to/EJ_3/folder$ make
    user@computer: path/to/EJ_3/folder$ ./run
\end{lstlisting}